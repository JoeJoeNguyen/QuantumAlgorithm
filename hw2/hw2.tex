\documentclass[12pt]{article}
\usepackage{amsmath}
\usepackage{amssymb}
\usepackage{geometry}
\geometry{a4paper, margin=1in}

\title{Homework 2 - Quantum Algorithms}
\author{Son Nguyen \\
I pledge my honor that I have abided by the Stevens Honor System.}
\date{\today}

\begin{document}

\maketitle
\section*{Problem 1}
\begin{itemize}
    \item \(\overline{(AB)} = \overline{A} \; \overline{B}\) \\
    Let \(A = \begin{bmatrix}
        a_{11} & a_{12} \\
        a_{21} & a_{22}
    \end{bmatrix}\) and \(B = \begin{bmatrix}
        b_{11} & b_{12} \\
        b_{21} & b_{22}
    \end{bmatrix}\). Then:
    \[A \cdot B = \begin{bmatrix}
        a_{11}b_{11} + a_{12}b_{21} & a_{11}b_{12} + a_{12}b_{22} \\
        a_{21}b_{11} + a_{22}b_{21} & a_{21}b_{12} + a_{22}b_{22}
    \end{bmatrix}\]
    \begin{align*}
        \overline{(AB)} &= \begin{bmatrix}
            \overline{a_{11}b_{11} + a_{12}b_{21}} & \overline{a_{11}b_{12} + a_{12}b_{22}} \\
            \overline{a_{21}b_{11} + a_{22}b_{21}} & \overline{a_{21}b_{12} + a_{22}b_{22}}
        \end{bmatrix}
    \end{align*}
    For the right hand side:
    \begin{align*}
        \overline{A} \; \overline{B} &= \begin{bmatrix}
            \overline{a_{11}} & \overline{a_{12}} \\
            \overline{a_{21}} & \overline{a_{22}}
        \end{bmatrix} \begin{bmatrix}
            \overline{b_{11}} & \overline{b_{12}} \\
            \overline{b_{21}} & \overline{b_{22}}
        \end{bmatrix} \\
        &= \begin{bmatrix}
            \overline{a_{11}b_{11}} + \overline{a_{12}b_{21}} & \overline{a_{11}b_{12}} + \overline{a_{12}b_{22}} \\
            \overline{a_{21}b_{11}} + \overline{a_{22}b_{21}} & \overline{a_{21}b_{12}} + \overline{a_{22}b_{22}}
        \end{bmatrix}\\
        &= \begin{bmatrix}
            \overline{a_{11}b_{11} + a_{12}b_{21}} & \overline{a_{11}b_{12} + a_{12}b_{22}} \\
            \overline{a_{21}b_{11} + a_{22}b_{21}} & \overline{a_{21}b_{12} + a_{22}b_{22}}
        \end{bmatrix}
    \end{align*}
    Since complex conjugation is an isomorphism. 
    \item \((AB)^T = B^T A^T\)
    \begin{align*}
        (AB)^T &= \begin{bmatrix}
            a_{11}b_{11} + a_{12}b_{21} & a_{11}b_{12} + a_{12}b_{22} \\
            a_{21}b_{11} + a_{22}b_{21} & a_{21}b_{12} + a_{22}b_{22}
        \end{bmatrix}^T \\
        &= \begin{bmatrix}
            a_{11}b_{11} + a_{12}b_{21} & a_{21}b_{11} + a_{22}b_{21} \\
            a_{11}b_{12} + a_{12}b_{22} & a_{21}b_{12} + a_{22}b_{22}
        \end{bmatrix} \\
        &= \begin{bmatrix}
            b_{11} & b_{21} \\
            b_{12} & b_{22}
        \end{bmatrix} \begin{bmatrix}
            a_{11} & a_{21} \\
            a_{12} & a_{22}
        \end{bmatrix} \\
        &= B^T A^T
    \end{align*}
\end{itemize}
\section*{Problem 2}
\[A = \begin{bmatrix}
    \frac{1+i}{2} & \frac{1}{\sqrt{3}} & \frac{3+i}{2\sqrt{15}} \\
    \frac{-1}{2} & \frac{1}{\sqrt{3}} & \frac{4 + 3i}{2\sqrt{15}} \\
    \frac{1}{2} & \frac{-i}{\sqrt{3}} & \frac{5i}{2\sqrt{15}}
\end{bmatrix}\]
A is unitary if \(A^{\dagger}A = I\).
\begin{align*}
    A^{\dagger}A &= \begin{bmatrix}
        \frac{1-i}{2} & \frac{-1}{2} & \frac{1}{2} \\
        \frac{1}{\sqrt{3}} & \frac{1}{\sqrt{3}} & \frac{i}{\sqrt{3}} \\
        \frac{3-i}{2\sqrt{15}} & \frac{4-3i}{2\sqrt{15}} & \frac{-5i}{2\sqrt{15}}
    \end{bmatrix} \begin{bmatrix}
        \frac{1+i}{2} & \frac{1}{\sqrt{3}} & \frac{3+i}{2\sqrt{15}} \\
        \frac{-1}{2} & \frac{1}{\sqrt{3}} & \frac{4 + 3i}{2\sqrt{15}} \\
        \frac{1}{2} & \frac{-i}{\sqrt{3}} & \frac{5i}{2\sqrt{15}}
    \end{bmatrix} \\
    &=  \begin{bmatrix}
        1 & \frac{-i}{\sqrt{3}} & 0 \\
        \frac{i}{\sqrt{3}} & 1 & \frac{(1+2i)\sqrt{5}}{15} \\
        0 & \frac{(1-2i)\sqrt{5}}{15} & 1
    \end{bmatrix}\\
    & \Rightarrow \text{A is not unitary}
\end{align*}

\end{document}