\documentclass[12pt]{article}
\usepackage{amsmath}
\usepackage{amssymb}
\usepackage{geometry}
\geometry{a4paper, margin=1in}

\title{Homework 3 - Quantum Algorithms}
\author{Son Nguyen \\
I pledge my honor that I have abided by the Stevens Honor System.}
\date{\today}

\begin{document}

\maketitle
\section{Problem 1}
\begin{itemize}
    \item It is deterministic turing machine because there is only one possible output for each input.
    \item 
    \begin{itemize}
        \item For \(\epsilon\) The machine starts with \(\_\), then it moves to \(q_1\), write 0 and moving left. Since \(q_1\) is the final state. The machine halts.
        \item For '000'
        \begin{align*}
            & \_\underbrace{0}_{q_0}00 \_ \\
            & \underbrace{\_}_{q_0} 000 \_ \\
            &  \underbrace{\_}_{q_1} 0000\_
        \end{align*}
        The machine halts.
        \item For '001'
        \begin{align*}
            &\_ \underbrace{0}_{q_0}01 \_ \\
            & \underbrace{\_}_{q_0} 001 \_ \\
            &  \underbrace{\_}_{q_1} 0001\_
        \end{align*}
        The machine halts.
        \item For '111'
        \begin{align*}
            &\_ \underbrace{1}_{q_0}11 \_ \\
            & \_ 1 \underbrace{1}_{q_0} 1\_ \\
            & \_ 11\underbrace{1}_{q_0} \_ \\
            & \_ 111\underbrace{\_}_{q_0} \\
            & \_11\underbrace{1}_{q_1}0 \_ 
        \end{align*}
        The machine halts.
        \item For '101'
        \begin{align*}
            &\_ \underbrace{1}_{q_0}01 \_ \\
            & \_ 1 \underbrace{0}_{q_0} 1\_ \\
            & \_ \underbrace{1}_{q_0}01 \_ \\
            & \_ 1 \underbrace{0}_{q_0} 1\_ \\
            & \dots
        \end{align*}
        The machines does not halt. 
    \end{itemize}
    \item The machine halts on inputs that does not starts with 1s and have 0s. Example:\\
    Set that accepted by the machine: \{010, 01010, \dots \} \\
    Set that rejected by the machine: \{1110, 1010, 1011, \dots \}
\end{itemize}
\section*{Problem 2}
\begin{itemize}
    \item The machine halts on all inputs. \(\Sigma = \{0,1\}\)
    \item If the length of the input is not modulo 3, the machine will adds 0 to the rightmost of the input else it will add 1 to the rightmost of the input and halts.
    From the transition functions, we can see that the machine will move left to right from state \(q_0 \rightarrow q_1 \rightarrow q_2 \rightarrow q_0\rightarrow \dots\) without changing the input string. If it reaches the end (blank) at state 
    \(q_0\), it will add 1 to the rightmost of the input and halts, else if it's state \(q_1\) or \(q_2\), it will add 0 to the rightmost of the input and halts.
    \\ \\
    For the string \(w\). If \(|w| \; \text{mod} \; 3 = 0\) then the machine will add 1 to the rightmost of the input and halts. If \(|w| \; \text{mod} \; 3 \neq 0\) then the machine will add 0 to the rightmost of the input and halts. 
\end{itemize}
\end{document}