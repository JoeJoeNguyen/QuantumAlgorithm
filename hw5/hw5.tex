\documentclass[12pt]{article}
\usepackage{amsmath}
\usepackage{amssymb}
\usepackage{geometry}
\usepackage{braket}
\geometry{a4paper, margin=1in}

\title{Homework 8 - Quantum Algorithms}
\author{Son Nguyen \\
I pledge my honor that I have abided by the Stevens Honor System.}
\date{\today}

\begin{document}

\maketitle
\section*{Problem 1}
\indent For the first circuit,  let state \(\ket{\psi} = \alpha \ket{0} + \beta \ket{1} \). After measurment, the state collapse to either \(p(\ket{0}) = |\alpha|^2\)
or \(p(\ket{1}) = |\beta|^2\). Applying U after the measurment give us either: \(p(U\ket{0}) = |\alpha|^2\) or \(p(U\ket{1}) = |\beta|^2\).
\\
\\
For the second circuit, we have \(\ket{\psi} =(\alpha \ket{0} + \beta \ket{1})  \otimes \ket{0}\). Applying \(\text{CNOT}_{01}\) give us:
\begin{align*}
    \text{CNOT}_{01}[(\alpha \ket{0} + \beta \ket{1}) \otimes \ket{0}] &= \text{CNOT}_{01} (\alpha \ket{00} + \beta \ket{10}) \\
    &= \alpha \ket{00} + \beta \ket{11}
\end{align*}
CNOT gate creates a second identical state on the second second qubit. Now we applying unitary \(U\) to the first qubit give us.
\begin{align*}
    (U \otimes I) (\alpha \ket{00} + \beta \ket{11}) &= \alpha(U \ket{0} \otimes \ket{0}) + \beta(U \ket{1} \otimes \ket{1})
\end{align*}
Now measuring the second qubit. If the second qubit is measured to be \(\ket{0}\), the state of the first qubit is \(U\ket{0}\) with probability \(|\alpha|^2\). If the second qubit is measured to be \(\ket{1}\), the state of the first qubit is \(U\ket{1}\) with probability \(|\beta|^2\).
\section*{Problem 2}

In the first circuit, after the measurment, we have state \(\ket{0}\). Applying \(U\) give us \(U\ket{0}\).
\\
For the second circuit, starting with state \(\ket{\psi} = \alpha \ket{0} + \beta \ket{1}\). Measurment will result in either \(\ket{0}\) or \(\ket{1}\). Applying SWAP gate to the state \(\ket{0}\) give us:
\begin{align*}
    \text{SWAP}(\ket{0} \otimes \ket{0}) &= \ket{0} \otimes \ket{0} \\
    &= \ket{0} \otimes \ket{0} = |00\rangle \\
    \text{SWAP}(\ket{0} \otimes \ket{1}) &= \ket{1} \otimes \ket{0} \\
    &= \ket{0} \otimes \ket{1} = |01\rangle
 \end{align*}
 Now applying \(U\) to the first qubit gives us: 
    \begin{align*}
        (U \otimes I) \ket{0} \otimes \ket{0} &= U\ket{0} \otimes \ket{0} \\
        (U \otimes I) \ket{0} \otimes \ket{1} &= U\ket{0} \otimes \ket{1}
    \end{align*}
No matter the coeficient \(\alpha\) and \(\beta\), we always have \(U\ket{0}\) as the state of the first qubit.
\section*{Problem 3}
For the first circuit: \\
Starting with \(\ket{\psi} = \alpha \ket{0} + \beta \ket{0}\) state, and \(\ket{\phi}\) state, after measurment on \(\ket{\psi}\) we have:
\begin{align*}
    p(\ket{0}) &= |\alpha|^2 \\
    p(\ket{1}) &= |\beta|^2
\end{align*}
Applying Control-\(U_{\psi,\phi}\) 
\begin{align*}
    &p(CU_{\psi,\theta}(\ket{0} \otimes \ket{\phi}) = \ket{0} \otimes \ket{\phi}) = |\alpha|^2 \\
    &p(CU_{\psi,\theta}(\ket{1} \otimes \ket{\phi}) = \ket{1} \otimes U\ket{\phi}) = |\beta|^2
\end{align*}
\\
For the second circuit: \\
\begin{align*}
    \ket{\psi} \otimes \ket{\phi} &= (\alpha \ket{0} + \beta \ket{1}) \otimes \ket{\phi} \\ 
    &= \alpha (\ket{0} \otimes \ket{\phi}) + \beta (\ket{1} \otimes \ket{\phi})
\end{align*}
Applying Control-\(U_{\psi,\phi}\):
\begin{align*}
    &CU_{\psi,\phi}[\alpha (\ket{0} \otimes \ket{\phi}) + \beta (\ket{1} \otimes \ket{\phi})]\\
    &= \alpha (\ket{0} \otimes \ket{\phi}) + \beta (\ket{1} \otimes U\ket{\phi})
\end{align*}
After measurment: 
\begin{align*}
    &p(\ket{0} \otimes \ket{\phi}) = |\alpha|^2 \\
    &p(\ket{1} \otimes U\ket{\phi}) = |\beta|^2
\end{align*}
\end{document}